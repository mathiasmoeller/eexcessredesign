% This is LLNCS.DEM the demonstration file of
% the LaTeX macro package from Springer-Verlag
% for Lecture Notes in Computer Science,
% version 2.4 for LaTeX2e as of 16. April 2010
%
\documentclass{llncs}
%
\usepackage{makeidx}  % allows for indexgeneration
\usepackage{graphicx}
\usepackage[ngerman]{babel}
\usepackage[utf8]{inputenc}
\usepackage{microtype}
\usepackage{enumitem}
%\setcounter{secnumdepth}{3}
\setcounter{tocdepth}{3}

%
\begin{document}
%
%\frontmatter          % for the preliminaries
%
\pagestyle{headings}  % switches on printing of running heads
%
%\mainmatter              % start of the contributions
%
\title{XY - ein \emph{Game with a Purpose}} 
%
\author{Lisa Wagner,\\
\email{wagnerli@fim.uni-passau.de}}
%
\institute{Universität Passau, 94032 Passau}

\maketitle              % typeset the title of the contribution

\pagestyle{plain}		% Zeilennummern unten Mitte


%
%\tableofcontents
%
\section{Zielsetzung} 
\label{sec:zielsetzung}
Ziel dieser Arbeit ist es, ein Online-Spiel zu entwickeln, das spielerisch kulturelle Inhalte aus der Europeana-Datenbank vermittelt. 
Im Spiel wird es zwei Arten von Usern geben: Eine Person stellt Fragen basierend auf Inhalten der Europeana-Datenbank und gibt zusätzlich Links zu den entsprechenden Ergebnis-Seiten an. Die User können selbst entscheiden, ob sie eine einzelne Frage mit einer einzelnen Antwort oder ob sie aufeinander aufbauende Fragen stellen möchten. Die Ergebnislinks müssen nicht unbedingt auf einen bestimmten Eintrag verweisen, auch der letzte Suchschritt kann ein korrektes Ergebnis darstellen.
Diese Frage-und-Antwort-Paare werden gesammelt und in einer Datenbank gespeichert. Der anderen Person wird eine der zur Verfügung stehenden Fragen gestellt. Sie muss dann versuchen, den korrekten Link innerhalb eines gewissen Zeitlimits durch gezielte Suchanfragen herauszufinden. Das Spiel ist gewonnen, wenn die Links übereinstimmen und das Zeitlimit noch nicht abgelaufen ist. \newline
Die verwendeten Suchbegriffe werden gespeichert, um erfolgreiche Lösungsstrategien auswerten zu können und so Einblicke ins Suchverhalten der User zu erhalten.


% section zielsetzung (end)

\section{Vorarbeit und Recherche}
%
Im Vorfeld der Arbeit wird untersucht, wie ein Spiel grundsätzlich aufgebaut sein sollte, um Spaß zu machen und zu motivieren. Auf dieser Basis wird skizziert wie das spätere Spiel aussehen und aufgebaut sein soll.
\newline
In diesem Schritt werden auch die nötigen Vorüberlegungen bezüglich Datenbankschema, Codestruktur und Basis-UI angestellt.
\newpage

\section{Benutzeroberfläche und Spielablauf}
Um hier einen ersten Eindruck vom geplanten Aufbau zu erhalten, wurde ein Entwurf der Oberfläche angefertigt.
\newline
\begin{figure}{}
\centering
\includegraphics[height=11cm]{images/home.png}
\caption{Erstentwurf der Startseite}
\end{figure}
\newline
Die User erhalten auf der Startseite einen ersten Überblick über die Spielregeln und haben die Möglichkeit direkt ein Spiel zu starten. Ein Klick auf \glqq Asking" führt zu einem Formular in das Frage und Antwort eingegeben und abgespeichert werden können. Ein Klick auf \glqq Answering" führt zunächst auf eine Seite, die die Regeln wie Zeitlimit, Suchen nur auf der Spiel-Seite erlaubt, etc. beschreibt. Durch einen weiteren Klick auf \glqq Start" wird die zufällig gewählte Frage aufgedeckt, die Suchmaske und der Timer angezeigt. Umgesetzt wird die Suchmaske durch die API der Europeana.

\newpage

\begin{figure}{}
\centering
\includegraphics[height=2cm]{images/signup_button.png}
\caption{User werden die Möglichkeit haben, sich im System zu registrieren}
\end{figure}
Über diesen Button werden nicht-registrierte User die Möglichkeit haben, ein Profil anzulegen. Um diesen Vorgang möglichst simpel und komfortabel zu gestalten, ist bisher geplant, \emph{OAuth2} zu verwenden.
\newline

\begin{figure}{}
\centering
\includegraphics[height=2cm]{images/signup.png}
\caption{Gründe für eine Registrierung}
\end{figure}
Auf der Startseite werden auch die Vorteile einer eventuellen Registrierung aufgelistet. Die Registrierung wird durch spezielle Features attraktiv gestaltet, da so die Zahl anonymer User verringert werden kann. Dies erhöht wiederum die Aussagekraft und Korrektheit der eingegeben Fragen und Antworten.



\newpage
\section{Überblick: Zeitlicher Ablauf und inhaltlicher Aufbau} % (fold)
\label{sec:zeitlicher_ablauf}
\begin{enumerate}[label=\Roman*)]
\item Recherche zu Motivation und nötigen Vorwissen für Implementierung
\item Skizzierung Programmaufbau, Datenbankschema, Interface
\item Implementierung:
	\begin{itemize}
	\item[]
	\begin{description} \itemsep5pt
		\item [Milestone 1:] Umsetzen des Datenbankschemas
		\item [Milestone 2:] Funktionierende Suche/Auslesen der Suchbegriffe
		\item [Milestone 3:] User-Profile (Registrierung, E-Mail-Verifikation)
		\item [Milestone 4:] Einstellen von Fragen und Antworten 
		\item [Milestone 5:] Funktionalität des eigentlichen Spiels
		\item [Milestone 6:] Zeitlimit, Generieren von Highscores, \emph{Show me how he did it}
		\item [Milestone 7:] Hilfe, Anleitung, evtl. weitere Gamification-Elemente
		\item [Milestone 8:] Usability Tests (optional)
		\item [Milestone 9:] Endgültiges Design (optional)\\


	\end{description}
\end{itemize}
\item Theoretischer Teil der Arbeit:
\begin{itemize}
	\item[]
	\begin{description} \itemsep5pt	
		\item [Einleitung] Beschreibung Europeana, Beschreibung GWAPs im Allgemeinen
		\item [Motivation] Beschreibung der Idee/Motivation der Arbeit (Verbindung von GWAP und Europeana)
		\item [Konzept] Beschreibung des Spielaufbaus, der Gamification-Elemente und der verwendeten Technologien
		\item [Implementierung] Beschreibung der Implementierungsphase
		\item [Evaluierung] Usability/Thinking aloud Tests
		\item [Community Building] Überlegungen zu möglicher (viraler) Bewerbung
\end{description}
\end{itemize}
\end{enumerate}
% section zeitlicher_ablauf (end)




\end{document}
