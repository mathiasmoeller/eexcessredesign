\section{Konzept}
 Wie auch bei der von EEXCESS entwickelten Extension fiel auch bei Jarvis die Wahl auf eine Chrome Extension. Google Chrome hat mit 40\% den größten Marktanteil unter Web Browsern und durch die Ähnlichkeit der Extension Architektur von Chrome zu anderen Browsern kann die Anwendung im Nachhinein leicht auf diese übertragen werden \cite{schlottererweb}. Per ``JavaScript Injection'' kann in diesen Architekturen das Aussehen und Verhalten der Seite verändert und neue Funktionalität hinzugefügt werden \cite{schlottererweb}.

 Beim Entwurf von Jarvis wurde sich am Design von JITIR-Agents orientiert. Trotzdem sucht Jarvis nicht proaktiv nach Ergebnissen, sondern erst nach einer Interaktion der Benutzerin. Zum einen entstehen durch die Nutzung der Europeana API technische Limitierungen. Für jeden API-Key lässt Europeana pro Tag nur 10.000 Anfragen zu \cite{europlimit}. Dieser API-Key wird von allen installierten Jarvis Extensions genutzt. Da in jeder geöffneten Webseite für jeden gefundenen Paragraphen eine eigene Anfrage an die Europeana API gesendet werden würde, wäre dieses Limit ab einer gewissen Anzahl von Nutzern schnell überschritten. Zum anderen kann die Nutzerin so entscheiden, ob er zum jeweiligen Paragraphen weitere Informationen erhalten möchte oder nicht. Möchte er das nicht, wird seine Aufmerksamkeit auch nicht durch das erscheinen von Ergebnissen gestört. Der letzte Grund, der zu dieser Entscheidung geführt hat, ist, dass das automatische Absenden aller Suchanfragen Rechenleistung benötigen würde, wodurch der Browser verlangsamt werden würde.

 Der Entwurf der Extension lässt sich in fünf Ziele einteilen:
 \begin{enumerate*}
 	\item Die Erstellung der Suchanfrage
 	\item Die Anzeige der Ergebnisse
  	\item Die Erklärung der Suchanfragen Generierung
 	\item Das Anpassen der Suchanfrage durch die Benutzerin
 	\item Die Verbesserung der Ergebnisgüte.
\end{enumerate*}
Auf diese Ziele wird im folgenden genauer eingegangen.

 \subsection{Erstellung der Suchanfrage}
 Um eine Suchanfrage abschicken zu können müssen erst Schlüsselwörter gefunden werden. Das geschieht entweder durch das analysieren des Paragraphen oder durch Hinzufügen von Wörtern durch die Benutzerin. Die Schlüsselwörter müssen daraufhin in eine Suchanfrage umgewandelt werden, die die Sprache der angesprochenen Datenbank verwendet. Im Falle von Europeana ist das zum Beispiel eine boolesche Abfragesprache. Zur eigentlichen Suchanfrage werden noch Suchparameter, wie maximale Anzahl der erwarteten Ergebnisse, hinzugefügt. Ist das geschehen wir sie abgeschickt.

 \subsection{Anzeige der Ergebnisse}
 Nachdem eine Anfrage an die Datenbank gesendet wurde und die Ergebnisse beim Client angekommen sind, müssen diese dargestellt werden. Die Nutzerin soll leicht erkennen können, welche der ihm angezeigten Elemente aus der ursprünglichen Webseite stammen und welche durch die Extension hinzugekommen sind. Aus diesem Grund wurde bei Jarvis ein einheitliches und buntes Design gewählt. So setzt sich die Anwendung deutlich von den meisten Webseiten ab. Auch soll eine enge Bindung zwischen den Ergebnissen und dem dazugehörigen Paragraphen geschaffen werden. Um das zu erreichen, werden die durch die Paragraphen-Erkennung gefundenen Textpassagen hervorgehoben und die Ergebnisse direkt bei den jeweiligen angezeigt.

 Weiterhin ist es wichtig, einen Mittelwert zu finden, was die Menge der angezeigten Informationen betrifft. Zu viele Informationen auf einmal können die Nutzerin ablenken. Auf der anderen Seite ist es das Ziel der Userin eine möglichst reichhaltige alternative Informationsquelle zu liefern \cite{rhodes2000margin}. Rhodes empfiehlt hierfür ein Ramping Interface \cite{rhodes2000just}. Hierbei werden die Informationen in Stufen aufgeteilt. Jede Stufe liefert etwas mehr Informationen als die vorherige. Auf diese Weise kann die Benutzerin entscheiden ob sie mehr Informationen sehen möchte und daraufhin auf die nächste Stufe gehen oder nicht.

 Jarvis zeigt nach einer erfolgreichen Suche zunächst nur die Anzahl der gefundenen Ergebnisse an. Diese Anzeige ist aufgeteilt in Anzahl der textuellen Ergebnisse, der gefundenen Bilder sowie der gefundenen Audio- und Videoquellen. Durch Auswählen einer der drei Kategorien werden in einem sich über dem Paragraphen öffnenden Fenster eine Liste mit Titel und Vorschaubild der Ergebnisse angezeigt. Die Liste ist absteigend nach der Relevanz sortiert, die Europeana in Form eines Relevance Scores mitliefert. Durch ändern des Tabs kann hier zwischen den Kategorien gewechselt werden.
 Durch einen Klick auf den Titel oder das Bild eines Listeneintrages kann die Userin die entsprechende Quelle aufrufen. Bewegt sie bei Text- oder Audio-/Videoquellen den Cursor über das Miniaturbild eines Ergebnisses, werden ihr weitere Informationen angezeigt. Da bei Bildquellen die Metadaten nicht so wichtig sind wie das Bild selber, werden sie in einer Gitter-Liste angezeigt. Die Listen-Elemente bestehen nur aus einem Vorschaubild sowie dem Titel. Durch klicken gelangt die Benutzerin zur Quelle des Bildes. Weitere Informationen werden nicht angezeigt, da sie für die Entscheidung die Quelle aufzurufen in den meisten Fällen unwichtig sind.

 \subsection{Erklärung der Suchanfragen Generierung}
 \label{ssec:suchanfragenGenerierung}
 Für die Userin soll es einfach zu erkennen sein, wie die Ergebnisse zustande gekommen sind. Dafür ist es wichtig, dass sie sieht aus welchen Suchbegriffen die Anfrage zusammengesetzt wurde. Nach dem Auslösen der Suchfunktion der Extension werden die Schlagwörter, die aus diesem Paragraphen extrahiert wurden, über diesem angezeigt. Diese Anzeige erlaubt auch die Manipulation der Suchanfrage. Weiterhin werden die gefundenen Schlagwörter im Text hervorgehoben. So soll gewährleistet werden, dass die Entstehung der Suchanfrage offensichtlich ist und die Nutzerin intuitiv erkennt, dass die Anfrage über die genannten Elemente angepasst werden kann.

 \subsection{Anpassen der Suchanfrage durch die Benutzerin}
 Da automatisch generierte Suchanfragen nicht immer die gewünschten Ergebnisse liefern, kann die Nutzerin die Suchanfrage selber erstellen oder die generierte manipulieren. Dazu kann sie Wörter aus dem Text hinzufügen oder eigene eingeben. Aus der bestehenden Anfrage kann sie unpassende oder mehrdeutige Suchbegriffe entfernen. Dazu wird die im vorigen Absatz beschriebene Anzeige der Suchbegriffe genutzt. Durch diese Interaktionsmöglichkeiten soll das Programm so intuitiv wie möglich gestaltet werden um Nutzer mit unterschiedlichen Erfahrungsleveln anzusprechen. Die beschriebenen Funktionalitäten können weiterhin für interaktive Elemente der Query Extension (siehe Kapitel \ref{sec:futureWork}) genutzt werden.

 \subsection{Verbesserung der Ergebnisgüte}
 Durch den Einsatz von Methoden des maschinellen Lernens sowie von Query Extension soll die Relevanz der Ergebnisse verbessert werden \cite{ruthven2003re}. Für diesen Anwendungsbereich geeignete Techniken sind Automatic Query Extension\footnote{Automatic Query Extension ist hierbei ein Beispiel für maschinelles Lernen.} (z.B. das Erstellen von Suchprofilen der User \cite{budzik2000user}) und Interactive Query Extension (z.B. Relevance Feedback \cite{harman1988towards}. Bei Query Extension werden, mit oder ohne Benutzerinteraktion, zusätzliche Begriffe zur Suchanfrage hinzugefügt. Da von Usern erstellte Suchanfragen meistens nur sehr kurz \cite{ruthven2003re} und dadurch oft nicht eindeutig sind, kann so eine Steigerung der Ergebnisgüte erreicht werden. Durch die hinzugefügten Suchwörter können falsche Ergebnisse, die durch andere Bedeutungen der alten Suchwörter entstanden sind, heraus gefiltert werden.

 Ein Beispiel für Automatic Query Extension ist das Erstellen von Suchprofilen. Das Programm versucht dabei vom Verhalten der Userin zu lernen. Dazu werden Informationen über alle Suchvorgänge hinweg gespeichert. Diese Informationen können durch das Bewerten von Seiten durch den Nutzer gesammelt werden oder aus den Seiten extrahiert werden, bei denen die Nutzerin ein Lesezeichen gesetzt hat \cite{budzik2000user}. Die gewonnenen Informationen werden dann in Schlüsselwörter umgewandelt und an die Suchanfragen angehängt.

 Interactive Query Expansion erlaubt der Userin, die initiale Anfrage noch zu erweitern \cite{harman1988towards}. Bei Relevance Feedback zum Beispiel werden der Benutzerin die Ergebnisse ihrer ersten Suchanfrage präsentiert. Sie kann daraufhin einzelne Quellen als relevant oder irrelevant bewerten. Auf Grund dieser Bewertung wird die ursprüngliche Anfrage um positive und negierte Terme erweitert \cite{budzik2000user}.

 Da das den Rahmen dieser Arbeit jedoch sprengen würde, wird auf die Integration dieser Methoden nur in Kapitel \ref{sec:futureWork} Bezug genommen.

  \subsubsection{Speicherung von guten Suchanfragen für jeden Paragraphen}
  \subsubsection{Bewertung von Ergebnissen}


 TODO: 
 - Speichern von guten Queries pro Paragraph
 - bewertung von queries (vie viele ergebnisse wurden angeschaut, wurden überhaupt ergebnisse angeschaut)
 - bewertung der ergebnisse
 - lernen was gute queries sind
