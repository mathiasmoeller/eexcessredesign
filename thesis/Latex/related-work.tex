\section{Related Work}
Rhodes \cite{rhodes2000just} definiert JITIR-Agents als eine Klasse von Programmen, die dem Benutzer weiterführende Informationen basierend auf seinem lokalen Kontext anzeigen. Dabei beschränkt sich der überwachte Kontext meist auf die virtuelle Umgebung des Benutzers, wie E-Mail, Webseiten und geöffnete Dokumente. Als Kerneigenschaften von JITIR-Agents nennt Rhodes Selbstständigkeit, die Fähigkeit Informationen in einer leicht zugänglichen und gleichzeitig unaufdringlichen Weise darzustellen und das Bewusstsein über den Kontext des Benutzers \cite{rhodes2000just}.
Das im Zuge dieser Bachelorarbeit entworfene Programm (im Folgenden als Jarvis bezeichnet) erfordert vom Nutzer die explizite Aufforderung um nach Informationen zu suchen. Gründe für diese Entscheidung werden im dritten Kapitel näher betrachtet. Trotz dieses Widerspruchs zu Rhodes Definition lässt sich Jarvis am besten mit dieser Klasse von Programmen vergleichen. Warum diese Klassifizierung zutrifft wird im folgenden Abschnitt beschrieben.

\subsection{Unterschiede und Gemeinsamkeiten zu JITIR-Agents}
Studien haben gezeigt, dass schon eine kleine Steigerung des Aufwands, der betrieben werden muss um eine Aufgabe zu erfüllen, dazu führen kann, dass man die Aufgabe gar nicht erst ausführt \cite{rhodes2000just}. Laut Miller reicht für die meisten Aufgaben eine Antwortverzögerung von mehr als zwei Sekunden aus, um die Nutzungshäufigkeit des dazugehörigen Programms zu vermindern \cite{miller1968response}. Längere Zeitintervalle erschweren es, den Kontext der gerade ausgeführten Aufgabe und die übergeordneten Aufgaben im Kurzeitgedächtnis zu behalten. Nun wird der Fall betrachtet, dass das Lesen einer Webseite unterbrochen wird, um eine Suche mit einer Suchmaschine durchzuführen. Die eigentliche Aufgabe behält der Nutzer im Kurzzeitgedächtnis. Je länger die Suche dauert und je mehr er sich dazu von seiner eigentlichen Arbeit distanzieren muss, desto schwerer wird es wieder zur Hauptaufgabe zurück zu kehren. Wenn der Exkurs zur Suchmaschine schwerer wiegt ist als die Güte der erwarteten Resultate wird die Suche nicht durchgeführt \cite{rhodes2000just}.

Dieses Problem wird versucht mit JITIR-Agents zu beheben. Durch ihre proaktive Arbeitsweise muss der Nutzer seine Tätigkeit nicht mehr unterbrechen, sondern nur noch entscheiden ob er weiter Informationen sehen möchte oder nicht \cite{rhodes2000just}. Beim Entwurf von Jarvis wurde entschieden, dass das Programm nicht völlig eigenständig nach Informationen sucht, sondern nur die Webseite in seine Paragraphen aufteilt. Der Nutzer kann dann entscheiden, ob er zu einem Paragraphen eine Suche durchführen möchte und diese dann per Klick starten. Wie bei einem JITIR-Agent wird die Suchanfrage automatisch aus den gewonnenen Kontextinformationen generiert. Da die Suche innerhalb weniger Millisekunden ausgelöst werden kann und sich der Nutzer dazu nicht von seiner eigentlichen Aufgabe distanzieren muss, bleibt die kognitive Belastung sehr gering. Allerdings entsteht auch so der Nachteil, den Jarvis mit JITIR-Agents gemein hat: Sie nutzen alle gefunden Informationen für ihre Suchanfragen und können nicht zwischen relevanten und unwichtigen Suchwörtern unterscheiden \cite{rhodes2000margin}. Automatisch gebaute Suchanfragen sind deshalb weniger exakt als Menschen-generierte \cite{rhodes2000just}. Um dem entgegen zu wirken hat der Benutzer von Jarvis im Nachhinein noch die Möglichkeit, die Suche anzupassen und erneut abzuschicken.

Die zweite von Rhodes beschriebene Eigenschaft von JITIR-Agents, die Fähigkeit Informationen in einer leicht zugänglichen und gleichzeitig unaufdringlichen Wiese darzustellen, hat auch beim Entwurf von Jarvis eine bedeutende Rolle gespielt. Für die Darstellung der Ergebnisse muss ein Mittelwert gefunden werden. Die zusätzlichen Elemente sollen den Benutzer nicht unnötig ablenken, allerdings will man die Seite um möglichst reichhaltige Informationen erweitern \cite{rhodes2000margin}. Wie diese Problematik im Falle von Jarvis gelöst wurde wird im Implementierungs-Teil dieser Arbeit beschrieben.

Jarvis analysiert die geöffnete Webseite und teilt diese in Paragraphen ein. Wie JITIR-Agents ist er sich also über den lokalen Kontext des Benutzers bewusst und wertet diesen aus. Die dritte Voraussetzung für JITIR-Agents ist somit erfüllt.

Trotz der eingeschränkten Selbstständigkeit lässt sich Jarvis folglich am besten mit denen von Rhodes beschriebenen Programmen vergleichen. Auf diese Erkenntnis aufbauend wird nachfolgend Jarvis mit existierenden Implementierungen von JITIR-Agents verglichen.

\subsection{Vergleich mit existierenden JITIR Agents}
 	\subsubsection{Remembrance Agent (in EMACS Editor)}
		Zeigt Quellen an auf Basis des geschriebenen Texts, Benutzer kann Suchanfrage dann auch noch manuell anpassen/verfeinern
		- > Vorteil von Suchmaschinen wird mit integriert
 	\subsubsection{Margin? Web Plugin ähnlich wie EEXCESS}
 	\subsubsection{Watson}
 	\subsubsection{EEXCESS?}
\subsection{Text Retrieval Algorithmen}
 		Term Frequency/Inverse Document Frequency algorithm,
 		Text rank
 \subsection{Unterschiede und Gemeinsamkeiten zu „Automatic help systems“ (z.B. Microsoft Office Assistant - > Domain spezifisch)}
 	Domain spezifisch vs. Domain unabhängig