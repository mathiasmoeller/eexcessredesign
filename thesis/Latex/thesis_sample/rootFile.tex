\documentclass[a4paper,12pt,titlepage, twoside, openright, cleardoubleempty]{scrreprt} %page-size, letter-size, with title, is an article

\usepackage[ngerman]{babel} %all inserted strings will be german
\usepackage{a4wide} %for better viewing of a4 pages
\usepackage{amsmath}
\usepackage{amssymb}

\usepackage{ntheorem}

\usepackage[T1]{fontenc} %T1-encoded fonts: auch W�rter mit Umlauten trennen
\usepackage{lmodern}
\usepackage[latin1]{inputenc} %letter-style like input � or � possible

\usepackage{vmargin} %Seitenr�nder einstellen leichtgemacht
\usepackage{fancyhdr} %definiere einfache Headings (mindestens V 1.99c notwendig)

\usepackage{float} %to be able to insert graphics with [H] Options
\usepackage{makeidx} %so the \printindex works

\usepackage{pst-all} %pretty good drawing kit

\usepackage{listings}

\usepackage[vlined,boxed]{algorithm2e}

\lstset{basicstyle=\ttfamily,breaklines=true}

\usepackage{remreset} %folgende Zeilen ben�tigt, damit Fu�noten sich nicht zur�cksetzen bei neuem Kapitel
\makeatletter
\@removefromreset{footnote}{chapter}
\makeatother 

\lstset{breaklines=true}
\lstset{basicstyle=\ttfamily}
\lstset{language=Java}
\lstset{tabsize=4}

\setcounter{secnumdepth}{3}% Numerierung auch f�r \subsubsection
\setcounter{tocdepth}{3}% nimm auch \subsubsections ins Inhaltsverz. auf

\clubpenalty = 10000
\widowpenalty = 10000
\displaywidowpenalty = 10000

\setpapersize{A4}
\setmarginsrb{3cm}{1cm}{3cm}{1cm}{6mm}{7mm}{5mm}{15mm}

%% Stil
\parindent 0cm                     % Absatzanfang wird nicht einger�ckt
\parskip1.5ex plus0.5ex minus0.5ex % Abstand zwischen zwei Abs�tzen

\newtheorem{Def}{Definition}

\pagestyle{fancy}
\renewcommand{\chaptermark}[1]{\markboth{\thechapter.\ #1}{}}
\fancyhf{} % clear all header and footer fields
\fancyhead[LE,RO]{{\headfont\thepage}} % left/right header for even/odd pages
\fancyhead[LO]{\headfont\nouppercase{\rightmark}} % header for left side (odd)
\fancyhead[RE]{\headfont\nouppercase{\leftmark}} % right header for even pages
\renewcommand{\headrulewidth}{0.5pt} % head rule
\renewcommand{\footrulewidth}{0pt} % no rule
% plainstyle
\fancypagestyle{plain}{%
\fancyhf{} % clear all header and footer fields
\renewcommand{\headrulewidth}{0pt}
\renewcommand{\footrulewidth}{0pt}
}

\usepackage[pdftex]{graphicx}
\usepackage[pdftex,
bookmarksnumbered,
bookmarks,
bookmarksopen,
bookmarksopenlevel=1,
hypertexnames,
breaklinks,
]{hyperref}
\hypersetup{
pdftitle    = {Titel},
pdfsubject  = {...},
pdfauthor   = {},
pdfkeywords = {},
colorlinks  = {false},
linkcolor   = {blue},
citecolor   = {cyan}
}
\pdfcompresslevel=9
\pdfinfo{
/CreationDate (D:2008 05 31 00 00 00) % year(4) month(2) day(2) hour(2) minute(2) second(2)
/ModDate      (D:2008 05 31 00 00 00) % modification date
}

%%%%%%%%%%%%%%%%%%%%%%%%%%%%%%%%%%%%%%
%%%%%%%%%%%%%%%%%%%%%%%%%%%%%%%%%%%%%%
%%%%%%%%%%%%%%%%%%%%%%%%%%%%%%%%%%%%%%
%%% Hier die richtige Trennung von W�rtern festlegen, die Latex sonst 
%%% falsch trennt.
%%%%%%%%%%%%%%%%%%%%%%%%%%%%%%%%%%%%%%
%%%%%%%%%%%%%%%%%%%%%%%%%%%%%%%%%%%%%%
%%%%%%%%%%%%%%%%%%%%%%%%%%%%%%%%%%%%%%
\hyphenation{Media-annotation}


\makeindex %must be before "begin{document}
\begin{document}
\graphicspath{{pics/}}

\pagestyle{empty}

\pagenumbering{roman}

%%%%%%%%%%%%%%%%%%%%%%%%%%%%%%%%%%%%%%
%%% Titelseite
%%%%%%%%%%%%%%%%%%%%%%%%%%%%%%%%%%%%%%
\titlehead{
Universit"at Passau\\
Fakult"at f"ur Informatik und Mathematik
}
\begin{titlepage}
\subject{type of work}
\title{any title}

\author{your name}


\date{\today}
%%% oder:
% \date{<schreib_irgendwas_rein>}


\publishers{
\normalsize
\begin{flushleft}
\begin{tabular}{ll}
\multicolumn{2}{l}{type of work}\\
\multicolumn{2}{l}{am Lehrstuhl f"ur verteilte Informationssysteme}\\
\multicolumn{2}{l}{der Fakult"at f"ur Informatik und Mathematik}\\
\multicolumn{2}{l}{der Universit"at Passau}\\
\\
%%%%%%%%%%%%%%%%%%%%%%%%%%%%%%%%
%%% Falls Diplomarbeit - bei BA nur ein Gutachter im Moment
%%%%%%%%%%%%%%%%%%%%%%%%%%%%%%%%
Erstgutachter: & Prof. Dr. Harald Kosch \\
Zweitgutachter: & name of prof \\
Betreuer: & Dipl. Inf. Florian Stegmaier
\end{tabular}
\end{flushleft}

%\includegraphics*[height=20pt]{logos/uni.jpg}
\hfill
%\includegraphics*[height=20pt]{logos/lehrstuhl.jpg}
}%end publishers

%\dedication{Widmung}
\end{titlepage}
\maketitle

\pagestyle{fancy}

%%%%%%%%%%%%%%%%%%%%%%%%%%%%%%%%%%%%%%
%%% Kurzfassung (wenn m�glich auf Englisch, ansonsten Deutsch)
%%%%%%%%%%%%%%%%%%%%%%%%%%%%%%%%%%%%%%
\chapter*{Kurzfassung}

This is an abstract.

%%%%%%%%%%%%%%%%%%%%%%%%%%%%%%%%%%%%%%
%%% Inhaltsverzeichnis
%%%%%%%%%%%%%%%%%%%%%%%%%%%%%%%%%%%%%%
\tableofcontents

\cleardoublepage %beginne neue Seitenz�hlung auf einer rechten Seite (ungerade Seitenzahl)

%%%%%%%%%%%%%%%%%%%%%%%%%%%%%%%%%%%%%%
%%% Tabellenverzeichnis
%%%%%%%%%%%%%%%%%%%%%%%%%%%%%%%%%%%%%%
\listoftables
\addcontentsline{toc}{chapter}{Tabellenverzeichnis}

%%%%%%%%%%%%%%%%%%%%%%%%%%%%%%%%%%%%%%
%%% Abbildungsverzeichnis
%%%%%%%%%%%%%%%%%%%%%%%%%%%%%%%%%%%%%%
\listoffigures
\addcontentsline{toc}{chapter}{\listfigurename}

%%%%%%%%%%%%%%%%%%%%%%%%%%%%%%%%%%%%%%
%%% Diese beiden Zeilen werden f�r Doppelseitigen Ausdruck ben�tigt und
%%% damit erst nach dem Inhaltsverzeichnis die Seiten gez�hlt werden.
%%%%%%%%%%%%%%%%%%%%%%%%%%%%%%%%%%%%%%
\cleardoublepage
\pagenumbering{arabic}

%%%%%%%%%%%%%%%%%%%%%%%%%%%%%%%%%%%%%%
%%% Ab hier beginnt die eigentliche Arbeit.
%%%
%%% 2 Varianten:
%%%     - entweder einzelne Kapitel mit include einbinden (bessere �bersicht)
%%%     - oder den ganzen Text einfach reinschreiben (un�bersichtlich)
%%% Wie es gemacht wird spielt f�r den eigentlichen Output keine Rolle.
%%%%%%%%%%%%%%%%%%%%%%%%%%%%%%%%%%%%%%
% ----------------------------------------------------------------------------------------------------------
% Einleitung
% ----------------------------------------------------------------------------------------------------------
\section{Einleitung und Motivation}
\subsection{Unterschiede Web Augmentation, Web Personalization, Web Customization}
WA is to the web what augmented reality is to the physical world

\subsection{Vormarsch von JITIR/Web Augmentation}
JITIR = Just-in-Time Information Retrieval
\begin{itemize}
	\item Möglichkeit das Web mit Funktionen anzureichern
	\item JITIR hat Vorteile/Nachteile gegenüber klassischen Suchmaschinen
		\begin{itemize}
			\item Vorteil Suchmaschinen: Wenn Benutzer klare Vorstellung von der Suche/ Suchanfrage hat oder genau weiß wonach er sucht
			\item Vorteil JITIR: Aktueller Task muss nicht komplett unterbrochen werden. Man verliert nicht den Überblick was man gerade macht/bleibt im Kurzzeitgedächtnis. \"JITIR Agents greatly reduce the cost of searching for information \"
			\item Nachteil JITIR: Ergebnisse sind nicht so exakt wie bei Suchmaschinen
		\end{itemize}
\end{itemize}

\subsection{Verbesserung des EEXCESS Plugins / Unterschiede zum EEXCESS Plugin}
Auswertung der alten Evaluierung?
Erleichtern von Recherchen durch einbinden von weiterführenden Links zu kulturellen Inhalten direkt in die betrachtete Webseite

\section{Related Work}
\subsection{Unterschiede und Gemeinsamkeiten zu JITIR-Agents}
 \subsubsection{Proaktiv vs. User Interaction}
 \subsubsection{Informationen darstellen in „Nonintrusive Manner“}
 \subsubsection{Awareness of user's local context}
\subsection{Vergleich mit existierenden JITIR Agents}
 	\subsubsection{Remembrance Agent (in EMACS Editor)}
		Zeigt Quellen an auf Basis des geschriebenen Texts, Benutzer kann Suchanfrage dann auch noch manuell anpassen/verfeinern
		→ Vorteil von Suchmaschinen wird mit integriert
 	\subsubsection{Margin? Web Plugin ähnlich wie EEXCESS}
 	\subsubsection{Watson}
 	\subsubsection{EEXCESS?}
 	\subsection{Text Retrieval Algorithmen}
 		Term Frequency/Inverse Document Frequency algorithm
 		Text rank
 \subsection{Unterschiede und Gemeinsamkeiten zu „Automatic help systems“ (z.B. Microsoft Office Assistant → Domain spezifisch)}
 	Domain spezifisch vs. Domain unabhängig

\section{Konzept}
 \subsection{Warum kein Proactiver JITIR-Agent?}
 	→ API Limitierung und decrease cognitive load 
	→ Benutzer entscheidet ob er weitere Informationen erhalten möchte
	→ Benutzer kann Suchanfrage erst anpassen (Nachteil von Margin Notes (Paper 4)
 \subsection{Anzeige der Ergebnisse}
 \subsection{Erklärung der Such-Anfragen Generierung}
 \subsection{Anpassen der Suchanfrage durch den Nutzer}
 \subsection{Verbesserung der Suchanfrage z.B. durch maschinelles Lernen}

\section{Implementierung}
 \subsection{Verwendung von AngularJS für alle Komponenten des Plugins}
 \subsection{Bau der GUIs}
		→ Darf den Benutzer nicht zu sehr ablenken
		→ Ergebnisse müssen in der Nähe ihrer „Quelle“ angezeigt werden (proximity compatibility pricinple)
		→ Benutzer muss klar zwischen Webseite und Augmentation unterscheiden können
		→ buntes, auffälliges Design
		→ Ramping interface: Mehr Benutzerinteraktionen führen zu mehr angezeigten Informationen (Erklärung der Stages) 
 \subsection{Einbindung der REST-Services}

\section{Evaluierung?}
Aufgaben die Benutzer mit EEXCESS Lösen mussten müssen sie jetzt mit Redesign lösen. Vergleich der Ergebnisse?

\section{Future Work}
 \subsection{Alternative Algorithmen zur Textanalyse}
 \subsection{Implementierung einer automatischen Suchanfragen-Verbesserung durch maschinelles Lernen}
 	\begin{itemize}
 		\item Mehr kontextuelle Informationen miteinbeziehen
 		\item Such-Profil des Nutzers erstellen
 	\end{itemize}
 \subsection{Verbesserung der Ergebniss-Güte}
	\begin{itemize}
		\item durch Query Expansion
 		\item durch Filtern der Ergebnisse (mehr Präzision da Ausbeute bei JITIR nicht so relevant)
	\end{itemize}
 \subsection{Anpassung der Anwendung auf mobile Nutzung}

 \section{Conclusion}
 \begin{itemize}
 		\item Steigerung der Effektivität und Produktivität von wissenschaftlichem Arbeiten
 		\item Starke Effektivitätssteigerung durch Punkte aus Future Work möglich?
 	\end{itemize}

%%%%%%%%%%%%%%%%%%%%%%%%%%%%%%%%%%%%%%
%%% Bibtex-Tool: http://jabref.sourceforge.net/
%%%%%%%%%%%%%%%%%%%%%%%%%%%%%%%%%%%%%%
\bibliographystyle{ieeetr}
\bibliography{bib.bib}
\addcontentsline{toc}{chapter}{\bibname}

\begin{center}

\begin{minipage}{.8\textwidth}

\chapter*{Eidesstattliche Erkl�rung}

Ich erkl�re hiermit, dass ich die vorliegende Arbeit selbst�ndig verfasst und
keine anderen als die angegebenen Quellen und Hilfsmittel verwendet habe.\\

Die Arbeit wurde bisher keiner anderen Pr�fungsbeh�rde vorgelegt und auch noch nicht ver�ffentlicht.\\

LOCATION, den DATE

\vspace{1cm}

yourName

\end{minipage}

\end{center}

\clearpage

\end{document}